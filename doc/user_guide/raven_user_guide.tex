%
% This is an example LaTeX file which uses the SANDreport class file.
% It shows how a SAND report should be formatted, what sections and
% elements it should contain, and how to use the SANDreport class.
% It uses the LaTeX article class, but not the strict option.
% ItINLreport uses .eps logos and files to show how pdflatex can be used
%
% Get the latest version of the class file and more at
%    http://www.cs.sandia.gov/~rolf/SANDreport
%
% This file and the SANDreport.cls file are based on information
% contained in "Guide to Preparing {SAND} Reports", Sand98-0730, edited
% by Tamara K. Locke, and the newer "Guide to Preparing SAND Reports and
% Other Communication Products", SAND2002-2068P.
% Please send corrections and suggestions for improvements to
% Rolf Riesen, Org. 9223, MS 1110, rolf@cs.sandia.gov
%
\documentclass[pdf,12pt]{INLreport}
% pslatex is really old (1994).  It attempts to merge the times and mathptm packages.
% My opinion is that it produces a really bad looking math font.  So why are we using it?
% If you just want to change the text font, you should just \usepackage{times}.
% \usepackage{pslatex}
\usepackage{times}
\usepackage[FIGBOTCAP,normal,bf,tight]{subfigure}
\usepackage{amsmath}
\usepackage{amssymb}
\usepackage{soul}
\usepackage{pifont}
\usepackage{enumerate}
\usepackage{listings}
\usepackage{fullpage}
\usepackage{xcolor}          % Using xcolor for more robust color specification
\usepackage{ifthen}          % For simple checking in newcommand blocks
\usepackage{textcomp}
\usepackage{mathtools}
\usepackage{relsize}
\usepackage{lscape}
\usepackage[toc,page]{appendix}
\usepackage{RAVEN}

\newtheorem{mydef}{Definition}
\newcommand{\norm}[1]{\lVert#1\rVert}
%\usepackage[table,xcdraw]{xcolor}
%\usepackage{authblk}         % For making the author list look prettier
%\renewcommand\Authsep{,~\,}

% Custom colors
\definecolor{deepblue}{rgb}{0,0,0.5}
\definecolor{deepred}{rgb}{0.6,0,0}
\definecolor{deepgreen}{rgb}{0,0.5,0}
\definecolor{forestgreen}{RGB}{34,139,34}
\definecolor{orangered}{RGB}{239,134,64}
\definecolor{darkblue}{rgb}{0.0,0.0,0.6}
\definecolor{gray}{rgb}{0.4,0.4,0.4}

\lstset {
  basicstyle=\ttfamily,
  frame=single
}


\setcounter{secnumdepth}{5}
\lstdefinestyle{XML} {
    language=XML,
    extendedchars=true,
    breaklines=true,
    breakatwhitespace=true,
%    emph={name,dim,interactive,overwrite},
    emphstyle=\color{red},
    basicstyle=\ttfamily,
%    columns=fullflexible,
    commentstyle=\color{gray}\upshape,
    morestring=[b]",
    morecomment=[s]{<?}{?>},
    morecomment=[s][\color{forestgreen}]{<!--}{-->},
    keywordstyle=\color{cyan},
    stringstyle=\ttfamily\color{black},
    tagstyle=\color{darkblue}\bf\ttfamily,
    morekeywords={name,type},
%    morekeywords={name,attribute,source,variables,version,type,release,x,z,y,xlabel,ylabel,how,text,param1,param2,color,label},
}
\lstset{language=python,upquote=true}

\usepackage{titlesec}
\newcommand{\sectionbreak}{\clearpage}
\setcounter{secnumdepth}{4}

%\titleformat{\paragraph}
%{\normalfont\normalsize\bfseries}{\theparagraph}{1em}{}
%\titlespacing*{\paragraph}
%{0pt}{3.25ex plus 1ex minus .2ex}{1.5ex plus .2ex}

%%%%%%%% Begin comands definition to input python code into document
\usepackage[utf8]{inputenc}

% Default fixed font does not support bold face
\DeclareFixedFont{\ttb}{T1}{txtt}{bx}{n}{9} % for bold
\DeclareFixedFont{\ttm}{T1}{txtt}{m}{n}{9}  % for normal

\usepackage{listings}

% Python style for highlighting
\newcommand\pythonstyle{\lstset{
language=Python,
basicstyle=\ttm,
otherkeywords={self, none, return},             % Add keywords here
keywordstyle=\ttb\color{deepblue},
emph={MyClass,__init__},          % Custom highlighting
emphstyle=\ttb\color{deepred},    % Custom highlighting style
stringstyle=\color{deepgreen},
frame=tb,                         % Any extra options here
showstringspaces=false            %
}}


% Python environment
\lstnewenvironment{python}[1][]
{
\pythonstyle
\lstset{#1}
}
{}

% Python for external files
\newcommand\pythonexternal[2][]{{
\pythonstyle
\lstinputlisting[#1]{#2}}}

\lstnewenvironment{xml}
{}
{}

% Python for inline
\newcommand\pythoninline[1]{{\pythonstyle\lstinline!#1!}}

% Named Colors for the comments below (Attempted to match git symbol colors)
\definecolor{RScolor}{HTML}{8EB361}  % Sonat (adjusted for clarity)
\definecolor{DPMcolor}{HTML}{E28B8D} % Dan
\definecolor{JCcolor}{HTML}{82A8D9}  % Josh (adjusted for clarity)
\definecolor{AAcolor}{HTML}{8D7F44}  % Andrea
\definecolor{CRcolor}{HTML}{AC39CE}  % Cristian
\definecolor{RKcolor}{HTML}{3ECC8D}  % Bob (adjusted for clarity)
\definecolor{DMcolor}{HTML}{276605}  % Diego (adjusted for clarity)
\definecolor{PTcolor}{HTML}{990000}  % Paul

\def\DRAFT{} % Uncomment this if you want to see the notes people have been adding
% Comment command for developers (Should only be used under active development)
\ifdefined\DRAFT
  \newcommand{\nameLabeler}[3]{\textcolor{#2}{[[#1: #3]]}}
\else
  \newcommand{\nameLabeler}[3]{}
\fi
\newcommand{\alfoa}[1] {\nameLabeler{Andrea}{AAcolor}{#1}}
\newcommand{\cristr}[1] {\nameLabeler{Cristian}{CRcolor}{#1}}
\newcommand{\mandd}[1] {\nameLabeler{Diego}{DMcolor}{#1}}
\newcommand{\maljdan}[1] {\nameLabeler{Dan}{DPMcolor}{#1}}
\newcommand{\cogljj}[1] {\nameLabeler{Josh}{JCcolor}{#1}}
\newcommand{\bobk}[1] {\nameLabeler{Bob}{RKcolor}{#1}}
\newcommand{\senrs}[1] {\nameLabeler{Sonat}{RScolor}{#1}}
\newcommand{\talbpaul}[1] {\nameLabeler{Paul}{PTcolor}{#1}}
% Commands for making the LaTeX a bit more uniform and cleaner
\newcommand{\TODO}[1]    {\textcolor{red}{\textit{(#1)}}}
\newcommand{\xmlAttrRequired}[1] {\textcolor{red}{\textbf{\texttt{#1}}}}
\newcommand{\xmlAttr}[1] {\textcolor{cyan}{\textbf{\texttt{#1}}}}
\newcommand{\xmlNodeRequired}[1] {\textcolor{deepblue}{\textbf{\texttt{<#1>}}}}
\newcommand{\xmlNode}[1] {\textcolor{darkblue}{\textbf{\texttt{<#1>}}}}
\newcommand{\xmlString}[1] {\textcolor{black}{\textbf{\texttt{'#1'}}}}
\newcommand{\xmlDesc}[1] {\textbf{\textit{#1}}} % Maybe a misnomer, but I am
                                                % using this to detail the data
                                                % type and necessity of an XML
                                                % node or attribute,
                                                % xmlDesc = XML description
\newcommand{\default}[1]{~\\*\textit{Default: #1}}
\newcommand{\nb} {\textcolor{deepgreen}{\textbf{~Note:}}~}


%%%%%%%% End comands definition to input python code into document

%\usepackage[dvips,light,first,bottomafter]{draftcopy}
%\draftcopyName{Sample, contains no OUO}{70}
%\draftcopyName{Draft}{300}

% The bm package provides \bm for bold math fonts.  Apparently
% \boldsymbol, which I used to always use, is now considered
% obsolete.  Also, \boldsymbol doesn't even seem to work with
% the fonts used in this particular document...
\usepackage{bm}


% Define tensors to be in bold math font.
\newcommand{\tensor}[1]{{\bm{#1}}}

% Override the formatting used by \vec.  Instead of a little arrow
% over the letter, this creates a bold character.
\renewcommand{\vec}{\bm}

% Define unit vector notation.  If you don't override the
% behavior of \vec, you probably want to use the second one.
\newcommand{\unit}[1]{\hat{\bm{#1}}}
% \newcommand{\unit}[1]{\hat{#1}}

% Use this to refer to a single component of a unit vector.
\newcommand{\scalarunit}[1]{\hat{#1}}

% \toprule, \midrule, \bottomrule for tables
\usepackage{booktabs}

% \llbracket, \rrbracket
\usepackage{stmaryrd}

\usepackage{hyperref}
\hypersetup{
    colorlinks,
    citecolor=black,
    filecolor=black,
    linkcolor=black,
    urlcolor=black
}

% Compress lists of citations like [33,34,35,36,37] to [33-37]
\usepackage{cite}

% If you want to relax some of the SAND98-0730 requirements, use the "relax"
% option. It adds spaces and boldface in the table of contents, and does not
% force the page layout sizes.
% e.g. \documentclass[relax,12pt]{SANDreport}
%
% You can also use the "strict" option, which applies even more of the
% SAND98-0730 guidelines. It gets rid of section numbers which are often
% useful; e.g. \documentclass[strict]{SANDreport}

% The INLreport class uses \flushbottom formatting by default (since
% it's intended to be two-sided document).  \flushbottom causes
% additional space to be inserted both before and after paragraphs so
% that no matter how much text is actually available, it fills up the
% page from top to bottom.  My feeling is that \raggedbottom looks much
% better, primarily because most people will view the report
% electronically and not in a two-sided printed format where some argue
% \raggedbottom looks worse.  If we really want to have the original
% behavior, we can comment out this line...
\raggedbottom
\setcounter{secnumdepth}{5} % show 5 levels of subsection
\setcounter{tocdepth}{5} % include 5 levels of subsection in table of contents

% ---------------------------------------------------------------------------- %
%
% Set the title, author, and date
%
\title{RAVEN User Guide}
%\author{%
%\begin{tabular}{c} Author 1 \\ University1 \\ Mail1 \\ \\
%Author 3 \\ University3 \\ Mail3 \end{tabular} \and
%\begin{tabular}{c} Author 2 \\ University2 \\ Mail2 \\ \\
%Author 4 \\ University4 \\ Mail4\\
%\end{tabular} }


\author{
\\Andrea Alfonsi
\\Cristian Rabiti
\\Diego Mandelli
\\Joshua Cogliati
\\Congjian Wang
\\Paul W. Talbot
\\Jia Zhou
\\Pralhad Burli
\\Mohammad G. Abdo
}
%Just people who actually ``developed'' a significant capability in the code should be placed here. Andrea
%\author{\textbf{\textit{Main Developers:}}  \\Andrea Alfonsi}
%\affil{Idaho National Laboratory, Idaho Falls, ID 83402}
%\\\{cristian.rabiti, andrea.alfonsi, joshua.cogliati, diego.mandelli, robert.kinoshita, ramazan.sen\}@inl.gov}

% There is a "Printed" date on the title page of a SAND report, so
% the generic \date should [WorkingDir:]generally be empty.
\date{}


% ---------------------------------------------------------------------------- %
% Set some things we need for SAND reports. These are mandatory
%
\SANDnum{INL/EXT-18-44465}
\SANDprintDate{\today}
\SANDauthor{Andrea Alfonsi, Cristian Rabiti, Diego Mandelli, Joshua Cogliati, Congjian Wang, Paul W. Talbot, Jia Zhou, Pralhad Burli,Mohammad G. Abdo}
\SANDreleaseType{Revision 1}


% ---------------------------------------------------------------------------- %
% Include the markings required for your SAND report. The default is "Unlimited
% Release". You may have to edit the file included here, or create your own
% (see the examples provided).
%
% \include{MarkOUO} % Not needed for unlimted release reports

\def\component#1{\texttt{#1}}

% ---------------------------------------------------------------------------- %
\newcommand{\systemtau}{\tensor{\tau}_{\!\text{SUPG}}}

% Added by Sonat
\usepackage{placeins}
\usepackage{array}

\newcolumntype{L}[1]{>{\raggedright\let\newline\\\arraybackslash\hspace{0pt}}m{#1}}
\newcolumntype{C}[1]{>{\centering\let\newline\\\arraybackslash\hspace{0pt}}m{#1}}
\newcolumntype{R}[1]{>{\raggedleft\let\newline\\\arraybackslash\hspace{0pt}}m{#1}}

% end added by Sonat
% ---------------------------------------------------------------------------- %
%
% Start the document
%

\begin{document}

    \maketitle

    % ------------------------------------------------------------------------ %
    % An Abstract is required for SAND reports
    %
%    \begin{abstract}
%    \input abstract
%    \end{abstract}


    % ------------------------------------------------------------------------ %
    % An Acknowledgement section is optional but important, if someone made
    % contributions or helped beyond the normal part of a work assignment.
    % Use \section* since we don't want it in the table of context
    %
%    \clearpage
%    \section*{Acknowledgment}



%	The format of this report is based on information found
%	in~\cite{Sand98-0730}.


    % ------------------------------------------------------------------------ %
    % The table of contents and list of figures and tables
    % Comment out \listoffigures and \listoftables if there are no
    % figures or tables. Make sure this starts on an odd numbered page
    %
    \cleardoublepage		% TOC needs to start on an odd page
    \tableofcontents
    %\listoffigures
    %\listoftables


    % ---------------------------------------------------------------------- %
    % An optional preface or Foreword
%    \clearpage
%    \section*{Preface}
%    \addcontentsline{toc}{section}{Preface}
%	Although muggles usually have only limited experience with
%	magic, and many even dispute its existence, it is worthwhile
%	to be open minded and explore the possibilities.


    % ---------------------------------------------------------------------- %
    % An optional executive summary
    %\clearpage
    %\section*{Summary}
    %\addcontentsline{toc}{section}{Summary}
    %\input{Summary.tex}
%	Once a certain level of mistrust and skepticism has
%	been overcome, magic finds many uses in todays science



%	and engineering. In this report we explain some of the
%	fundamental spells and instruments of magic and wizardry. We
%	then conclude with a few examples on how they can be used
%	in daily activities at national Laboratories.


    % ---------------------------------------------------------------------- %
    % An optional glossary. We don't want it to be numbered
%    \clearpage
%    \section*{Nomenclature}
%    \addcontentsline{toc}{section}{Nomenclature}
%    \begin{description}
%          \item[alohomoral]
%           spell to open locked doors and containers
%          \item[leviosa]
%           spell to levitate objects
%    \item[remembrall]
%           device to alert you that you have forgotten something
%    \item[wand]
%           device to execute spells
%    \end{description}


    % ---------------------------------------------------------------------- %
    % This is where the body of the report begins; usually with an Introduction
    %
    \SANDmain		% Start the main part of the report

\section{Introduction}

\subsection{Project Background}
The development of RAVEN started in 2012 when, within the Nuclear Energy
Advanced Modeling and Simulation (NEAMS) program~\cite{neams}, the need of a modern
risk evaluation framework arose.
RAVEN's principal assignment is to provide the necessary software and algorithms
in order to employ the concepts developed by the Risk Informed Safety Margin
Characterization (RISMC) Pathway.
RISMC is one of the pathways defined within the Light Water Reactor
Sustainability (LWRS) program~\cite{lwrs}.

The goal of the RISMC approach is  the identification not only of the frequency of an
event which can potentially lead to system failure, but also the proximity (or lack
thereof) to key safety-related events: the safety margin.
Hence, the approach is interested in identifying and increasing the safety
margins related to those events.
A safety margin is a numerical value quantifying the probability that a safety
metric (e.g. peak pressure in a pipe) is exceeded under certain conditions.
% Conclusion
Most of the capabilities, implemented having Reactor Excursion and Leak Analysis Program v.7
(RELAP-7) as a principal focus, are
easily deployable to other system codes.
%
For this reason, several side activates have been employed (e.g.  RELAP5-3D~\cite{RELAP5userManual}, any Multiphysics Object Oriented
Simulation Environment-based App, etc.)
or are currently ongoing for coupling RAVEN with several different software.
%

\subsection{Acquiring and Installing RAVEN}
RAVEN is supported on three separate computing platforms: Linux, OSX (Apple Macintosh), and Microsoft Windows.
Currently, RAVEN is open-source and downloadable from RAVEN GitHub repository: \url{https://github.com/idaholab/raven}.
New users should visit \url{https://github.com/idaholab/raven/wiki} or refer to the user manual ~\cite{RAVENuserManual}
to get started with RAVEN. This typically involves the following steps:
\begin{itemize}
  \item \textit{Download RAVEN}
    \\ You can download the source code of RAVEN from \url{https://github.com/idaholab/raven}.
  \item \textit{Install RAVEN dependencies}
    \\ Instructions are available from \url{https://github.com/idaholab/raven/wiki}, or the user manual ~\cite{RAVENuserManual}.
  \item \textit{Install RAVEN}
    \\ Instructions are available from \url{https://github.com/idaholab/raven/wiki}, or the user manual ~\cite{RAVENuserManual}.
  \item \textit{Run RAVEN}
    \\ If RAVEN is installed successfully, please run the regression tests to verify your installation:
    \begin{lstlisting}[language=bash]
      ./run_tests
    \end{lstlisting}
    Normally there are skipped tests because either some of the codes are not available, or some of the test are not
    currently working. The output will explain why each is skipped. If all the tests pass, you are ready to run RAVEN.
    Now, open a terminal and use the following command (replace \texttt{<inputFileName.xml>} with your
    RAVEN input file):
    \begin{lstlisting}[language=bash]
      raven_framework <inputFileName.xml>
    \end{lstlisting}
    where the \texttt{raven\_framework} script can be found in the RAVEN folder. Alternatively, the \texttt{raven\_framework.py} script
    contained in the folder ``\texttt{raven}'' can be directly used:
    \begin{lstlisting}[language=bash]
      python raven/raven_framework.py <inputFileName.xml>
    \end{lstlisting}
  \item \textit{Participate in RAVEN user communities}
    \\ Join RAVEN mail lists to get help and updates of RAVEN: \url{https://groups.google.com/forum/#!forum/inl-raven-users}.
\end{itemize}

\input{userGuideFormats.tex}

\subsection{Capabilities of RAVEN}
% High-level RAVEN description
RAVEN~\cite{alfonsiMC} ~\cite{alfonsiPSA}~\cite{RAVENFY13}~\cite{ESREL2014} is a software framework that allows the user to perform parametric and stochastic
analysis based on the response of complex system codes.
The initial development was designed to provide dynamic probabilistic risk analysis
capabilities (DPRA) to the thermal-hydraulic code RELAP-7~\cite{relap7FY12}, currently under development
at Idaho National Laboratory (INL).
Now, RAVEN is not only a framework to perform DPRA but it is a flexible and
multi-purpose uncertainty quantification, regression analysis, probabilistic risk assessment, data analysis and
model optimization platform. Depending on the tasks to be accomplished and on the probabilistic characterization
of the problem, RAVEN perturbs (e.g., Monte-Carlo, Latin hypercube, reliability surface search) the response of
the system under consideration by altering its own parameters. The system is modeled by third party software
(e.g., RELAP5-3D, MAAP5, BISON, etc.) and accessible to RAVEN either directly (software coupling) or indirectly
(via input/output files). The data generated by the sampling process is analyzed using classical statistical
and more advanced data mining approaches. RAVEN also manages the parallel dispatching
(i.e. both on desktop/workstation and large High Performance Computing machines) of the software representing the
physical model. RAVEN heavily relies on artificial intelligence algorithms to construct surrogate models of
complex physical systems in order to perform uncertainty quantification, reliability analysis (limit state surface)
and parametric studies.

The main capabilities of RAVEN, with brief descriptions, are summarized here, or one can check the Figure.~\ref{fig:ravenCap}.
These capabilities may be used on their own or as building blocks to construct the sought workflow. In addition, RAVEN also provides some more sophisticated
\textbf{Ensemble algorithms} such as \textbf{EnsembleForward}, \textbf{EnsembleModel} to combine the existing
capabilites.

\begin{figure}[h!]
  \includegraphics[width=\textwidth]{pics/raven_cap.png}
  \caption{RAVEN Capabilities vs. Needs}
  \label{fig:ravenCap}
\end{figure}

\begin{itemize}
  \item \textbf{Sensitivity Analysis and Uncertainty Quantification}: Sensitivity analysis is a mathematical tool
    that can be used to identify the key sources of uncertainties. Uncertainty quantification is a process by which
    probabilistic information about system responses can be computed according to specified input parameter probability
    distributions. Available approaches in RAVEN include \textbf{Monte Carlo}, \textbf{Grid}, \textbf{Stratified (Latin hypercube)},
    \textbf{Sparse Grid Collocation}, \textbf{Sobol}, \textbf{Adaptive Sparse Grid}, \textbf{Adaptive Sobol} and
    \textbf{BasicStatistics}.
  \item \textbf{Design of Experiments}: The design of experiments (DOE) is a powerful tool that can be used to explore the
    parameter space at a variety of experimental situations. It can be used to determine the relationship between
    input factors and the desired outputs. Available approaches in RAVEN include \textbf{Factorial Design}
    (i.e. General full factorial, 2-level fractional-factorial and Plackett-Burman) and \textbf{Response Surface Design}
    (i.e. Box-Behnken and Central composite algorithms).
  \item \textbf{Risk Mitigation or Model Optimization}: RAVEN uses the \textbf{Optimizer}, a powerful sampler-like entity that searches
    the input space to find minimum or maximum values of a reponse. Currently available optimizers include
    \textbf{Simultaneous Perturbation Stochastic Approximation (SPSA)}.
  \item \textbf{Risk Analysis}: Available approaches in RAVEN include \textbf{Dynamic Event Tree}, \textbf{Limit Surface Search},
    \textbf{Hybrid Dynamic Event Tree}, \textbf{Adaptive Dynamic Event Tree}, \textbf{Adaptive Hybrid Dynamic Event Tree},
    \textbf{Data Mining}, \textbf{Importance Rank}, \textbf{Safest Point}, and \textbf{Basic Statistics}.
  \item \textbf{Risk Management}: Available approaches in RAVEN include \textbf{Reduced order models},
    approaches used for sensitivity and uncertainty analysis, and \textbf{Dynamic Event Tree} methods.
  \item \textbf{Validation}: Available approaches in RAVEN include \textbf{ROMs}, \textbf{Comparison Statistics} and \textbf{Validation Metrics} 
\end{itemize}

In addition, RAVEN includes a number of related advanced capabilities. \textbf{Surrogate or Reduced order models (ROMs)} are mathematical
model trained to predict a response of interest of a physical system. Typically, ROMs trade speed for accuracy representing
a faster, rough estimate of the underlying systems. They can be used to explore the input parameter space for optimization
or sensitivity and uncertainty studies. \textbf{Ensemble Model} is able to combine \textbf{Codes}, \textbf{External Models}
and \textbf{ROMs}. It is intended to create a chain of models whose execution order is determined by the input/output
relationships among them. If the relationships among the models evolve in a non-linear system, a Picard's iteration scheme is
employed.

\input{ravenComponents.tex}

\subsection{Code Interfaces of RAVEN}
The procedure of coupling a new code/application with RAVEN is a straightforward process. The provided Application
Programming Interfaces (APIs) allow RAVEN to interact with any code as long as all the parameters that need to be
perturbed are accessible by input files or via python interfaces. For example, for all the codes currently
supported by RAVEN (e.g. RELAP-7, RELAP-5D, BISON, MAMMOTH, etc.), the coupling is performed through a Python interface
that interprets the information coming from RAVEN and translates them into the input of the driven code. The couping procedure
does not require modifying RAVEN itself. Instread, the developer creates a new Python interface that is going to
be embedded in RAVEN at run-time (no need to introduce hard-coded coupling statements). In addition, RAVEN will
manage concurrent executions of your simulations in parallel, whether on a local desktop or remote high-performance cluster.

Figure.~\ref{fig:modelAPIs} depicts the different APIs between RAVEN and the computational models, i.e. the
\textbf{ROM}, \textbf{External Models} and \textbf{External Code} APIs. 

\begin{figure}[h!]
  \includegraphics[width=\textwidth]{pics/modelAPIs.png}
  \caption{RAVEN Application Programming Interfaces}
  \label{fig:modelAPIs}
\end{figure}

\clearpage

\subsection{User Guide Organization}
The goal of this document is to provide a set of detailed examples that can help the user to become familiar with
the RAVEN code. RAVEN is capable of investigating system response and explore input space using various
sampling schemes such as Monte Carlo, grid, or Latin Hypercube. However, RAVEN strength lies in its system feature
discovery capabilities such as: constructing limit surfaces, separating regions of the input space leading to
system failure, and using dynamic supervised learning techniques. New users should consult the \textbf{RAVEN Tutorial}
to get started.

\begin{itemize}
  \item \textbf{RAVEN Tutorial}: section~\ref{sec:ravenTutorial}
  \item \textbf{Sampling Strategies}: section~\ref{sec:forwardSamplingStrategies} and section~\ref{sec:adaptiveSamplingStrategies}
  \item \textbf{Restart}: section~\ref{sec:samplingFromRestart}
  \item \textbf{Reduced Order Modeling}: section~\ref{sec:ROMraven}
  \item \textbf{Risk Analysis}: section~\ref{sec:SAraven}
  \item \textbf{Data Mining}: section~\ref{sec:DMraven}
  \item \textbf{Model Optimization}: section~\ref{sec:optimizerStrategies}
  %%%% We may need to add the following sections
  %\item \textbf{Sensitivity Analysis and Uncertainty Quantification}:
  %\item \textbf{Design of Experiments}:
  %\item \textbf{Validation}:
  %\item \textbf{Code Interface}:
  %\item \textbf{Advanced Topics}:
  %  \begin{itemize}
  %    \item Ensemble Forward Sampler:
  %    \item Ensemble Model:
  %  \end{itemize}
\end{itemize}




\input{ravenTutorial.tex}
\input{forwardSamplingExample.tex}
\input{adaptiveSamplingExample.tex}
\input{restartSampling.tex}
\input{reducedOrderModelingExample.tex}
\input{statisticalAnalysisExample.tex}
\input{dataMiningExample.tex}
\input{optimizing.tex}
\input{ensembleModel.tex}
\clearpage
\begin{appendices}
  %\section{Running RAVEN}
\label{HowToRun}

% I don't think this is mentioned earlier? Andrea answers :D It mentioned in the Introduction
%As already mentioned,
The RAVEN code is a blend of C++, C, and Python software. The entry point
resides on the Python side and is accessible via a command line interface.
%
After following the instructions in the previous Section, RAVEN is ready to be
used.
%
The \texttt{raven\_framework} script is in the raven folder.
%
To run RAVEN, open a terminal and use the following command (replace \texttt{<inputFileName.xml>} with your RAVEN input file):

\begin{itemize}

  \item \textbf{Any unix-based systems (e.g. Macintosh, Linux, etc.)}:
\begin{lstlisting}[language=bash]
raven_framework <inputFileName.xml>
\end{lstlisting}
  \item \textbf{Windows}:
  \begin{lstlisting}[language=bash]
bash.exe raven_framework <inputFileName.xml>
\end{lstlisting}
  
\end{itemize}

Using \texttt{raven\_framework} is the recommended way to run RAVEN.  In the event bypassing the typical
environment loading and checks is desired, it can also be run via
the \texttt{raven\_framework.py} script using python, with the input file as argument.  However, this is not
recommended, as it will use whatever default versions of Python and other libraries are discovered, rather
than the matching libraries set up during installation.

\nb For Windows systems, we provided a convenient Batch script ( \texttt{raven\_framework.bat} ) for running RAVEN 
avoiding to interact with the Windows command line terminal. More info on how to use it can be found in the RAVEN
\wiki , section \textit{Running RAVEN} (\url{https://github.com/idaholab/raven/wiki/runningRAVEN}).


  \section{Document Version Information}
  \input{../version.tex}
\end{appendices}
%\appendix
\section{Appendix: Example Primer}
\label{sec:examplePrimer}
In this Appendix, a set of examples are reported. In order to be as general as possible, the \textit{Model} type ``ExternalModel'' has been used.
%%%% EXAMPLE 1
\subsection{Example 1.}
\label{subsec:ex1}
This simple example is about the construction of a ``Lorentz attractor'', sampling the relative input space. The parameters that are sampled represent the initial coordinate (x0,y0,z0) of the attractor origin.

\begin{lstlisting}[style=XML,morekeywords={debug,re,seeding,class,subType,limit}]
<?xml version="1.0" encoding="UTF-8"?>
<Simulation verbosity="debug">
<!-- RUNINFO -->
<RunInfo>
    <WorkingDir>externalModel</WorkingDir>
    <Sequence>FirstMRun</Sequence>
    <batchSize>3</batchSize>
</RunInfo>
<!-- Files -->
<Files>
    <Input name='lorentzAttractor.py' type=''>lorentzAttractor</Input>
</Files>
<!-- STEPS -->
<Steps>
    <MultiRun name='FirstMRun'  re-seeding='25061978'>
        <Input   class='Files'     type=''               >lorentzAttractor.py</Input>
        <Model   class='Models'    type='ExternalModel'  >PythonModule</Model>
        <Sampler class='Samplers'  type='MonteCarlo'     >MC_external</Sampler>
        <Output  class='DataObjects'     type='HistorySet'      >testPrintHistorySet</Output>
        <Output  class='Databases' type='HDF5'           >test_external_db</Output>
        <Output  class='OutStreams' type='Print'   >testPrintHistorySet_dump</Output>
    </MultiRun >
</Steps>
<!-- MODELS -->
<Models>
    <ExternalModel name='PythonModule' subType='' ModuleToLoad='externalModel/lorentzAttractor'>
       <variables>sigma,rho,beta,x,y,z,time,x0,y0,z0</variables>
    </ExternalModel>
</Models>
<!-- DISTRIBUTIONS -->
<Distributions>
    <Normal name='x0_distrib'>
        <mean>4</mean>
        <sigma>1</sigma>
    </Normal>
    <Normal name='y0_distrib'>
        <mean>4</mean>
        <sigma>1</sigma>
    </Normal>
    <Normal name='z0_distrib'>
        <mean>4</mean>
        <sigma>1</sigma>
    </Normal>
</Distributions>
<!-- SAMPLERS -->
<Samplers>
    <MonteCarlo name='MC_external'>
      <samplerInit>
        <limit>3</limit>
      </samplerInit>
      <variable name='x0' >
        <distribution  >x0_distrib</distribution>
      </variable>
      <variable name='y0' >
        <distribution  >y0_distrib</distribution>
      </variable>
      <variable name='z0' >
        <distribution  >z0_distrib</distribution>
      </variable>
    </MonteCarlo>
</Samplers>
<!-- DATABASES -->
<Databases>
  <HDF5 name="test_external_db"/>
</Databases>
<!-- OUTSTREAMS -->
<OutStreams>
  <Print name='testPrintHistorySet_dump'>
    <type>csv</type>
    <source>testPrintHistorySet</source>
  </Print>
</OutStreams>
<!-- DATA OBJECTS -->
<DataObjects>
    <HistorySet name='testPrintHistorySet'>
        <Input>x0,y0,z0</Input>
        <Output>time,x,y,z</Output>
   </HistorySet>
</DataObjects>
</Simulation>
\end{lstlisting}
The Python \textit{ExternalModel} is reported below:
\begin{lstlisting}[language=python]
import numpy as np

def run(self,Input):
  max_time = 0.03
  t_step = 0.01

  numberTimeSteps = int(max_time/t_step)

  self.x = np.zeros(numberTimeSteps)
  self.y = np.zeros(numberTimeSteps)
  self.z = np.zeros(numberTimeSteps)
  self.time = np.zeros(numberTimeSteps)

  self.x0 = Input['x0']
  self.y0 = Input['y0']
  self.z0 = Input['z0']

  self.x[0] = Input['x0']
  self.y[0] = Input['y0']
  self.z[0] = Input['z0']
  self.time[0]= 0

  for t in range (numberTimeSteps-1):
    self.time[t+1] = self.time[t] + t_step
    self.x[t+1]    = self.x[t] +  self.sigma*
                      (self.y[t]-self.x[t]) * t_step
    self.y[t+1]    = self.y[t] + (self.x[t]*
                      (self.rho-self.z[t])-self.y[t]) * t_step
    self.z[t+1]    = self.z[t] + (self.x[t]*
                          self.y[t]-self.beta*self.z[t]) * t_step
\end{lstlisting}
%%%% EXAMPLE 2
\subsection{Example 2.}
\label{subsec:ex1}
This example shows a slightly more complicated example, that employs the usage of:
\begin{itemize}
    \item \textit{Samplers:} Grid and Adaptive;
    \item \textit{Models:} External, Reduce Order Models and Post-Processors;
    \item \textit{OutStreams:} Prints and Plots;
    \item \textit{Data Objects:} PointSets;
    \item \textit{Functions:} ExternalFunctions.
\end{itemize}
The goal of this input is to compute the ``SafestPoint''.
It provides the coordinates of the farthest
point from the limit surface that is given as an input.
%
The safest point coordinates are expected values of the coordinates of the
farthest points from the limit surface in the space of the ``controllable''
variables based on the probability distributions of the ``non-controllable''
variables.

The term ``controllable'' identifies those variables that are under control
during the system operation, while the ``non-controllable'' variables are
stochastic parameters affecting the system behavior randomly.

The ``SafestPoint'' post-processor requires the set of points belonging to the
limit surface, which must be given as an input.

\begin{lstlisting}[style=XML,morekeywords={debug,re,seeding,class,subType,limit}]
<Simulation verbosity='debug'>

<!-- RUNINFO -->
<RunInfo>
  <WorkingDir>SafestPointPP</WorkingDir>
  <Sequence>pth1,pth2,pth3,pth4</Sequence>
  <batchSize>50</batchSize>
</RunInfo>

<!-- STEPS -->
<Steps>
  <MultiRun name = 'pth1' pauseAtEnd = 'False'>
    <Sampler  class = 'Samplers'  type = 'Grid'           >grd_vl_ql_smp_dpt</Sampler>
    <Input    class = 'DataObjects'     type = 'PointSet'   >grd_vl_ql_smp_dpt_dt</Input>
    <Model    class = 'Models'    type = 'ExternalModel'  >xtr_mdl</Model>
    <Output   class = 'DataObjects'     type = 'PointSet'   >nt_phy_dpt_dt</Output>
  </MultiRun >

  <MultiRun name = 'pth2' pauseAtEnd = 'True'>
    <Sampler          class = 'Samplers'  type = 'Adaptive'      >dpt_smp</Sampler>
    <Input            class = 'DataObjects'     type = 'PointSet'  >bln_smp_dt</Input>
    <Model            class = 'Models'    type = 'ExternalModel' >xtr_mdl</Model>
    <Output           class = 'DataObjects'     type = 'PointSet'  >nt_phy_dpt_dt</Output>
    <SolutionExport   class = 'DataObjects'     type = 'PointSet'  >lmt_srf_dt</SolutionExport>
  </MultiRun>

  <PostProcess name='pth3' pauseAtEnd = 'False'>
    <Input    class = 'DataObjects'          type = 'PointSet'       >lmt_srf_dt</Input>
    <Model    class = 'Models'         type = 'PostProcessor'  >SP</Model>
    <Output   class = 'DataObjects'          type = 'PointSet'     >sfs_pnt_dt</Output>
  </PostProcess>

  <OutStreamStep name = 'pth4' pauseAtEnd = 'True'>
  	<Input  class = 'DataObjects'            type = 'PointSet'  >lmt_srf_dt</Input>
  	<Output class = 'OutStreams' type = 'Print'         >lmt_srf_dmp</Output>
    <Input  class = 'DataObjects'            type = 'PointSet'  >sfs_pnt_dt</Input>
  	<Output class = 'OutStreams' type = 'Print'         >sfs_pnt_dmp</Output>
  </OutStreamStep>
</Steps>

<!-- DATA OBJECTS -->
<DataObjects>
  <PointSet name = 'grd_vl_ql_smp_dpt_dt'>
    <Input>x1,x2,gammay</Input>
    <Output>OutputPlaceHolder</Output>
  </PointSet>

  <PointSet name = 'nt_phy_dpt_dt'>
    <Input>x1,x2,gammay</Input>
    <Output>g</Output>
  </PointSet>

  <PointSet name = 'bln_smp_dt'>
    <Input>x1,x2,gammay</Input>
    <Output>OutputPlaceHolder</Output>
  </PointSet>

  <PointSet name = 'lmt_srf_dt'>
    <Input>x1,x2,gammay</Input>
    <Output>g_zr</Output>
  </PointSet>

  <PointSet name = 'sfs_pnt_dt'>
    <Input>x1,x2,gammay</Input>
    <Output>p</Output>
  </PointSet>
</DataObjects>

<!-- DISTRIBUTIONS -->
<Distributions>
  <Normal name = 'x1_dst'>
    <upperBound>10</upperBound>
    <lowerBound>-10</lowerBound>
  	<mean>0.5</mean>
    <sigma>0.1</sigma>
  </Normal>

  <Normal name = 'x2_dst'>
    <upperBound>10</upperBound>
    <lowerBound>-10</lowerBound>
    <mean>-0.15</mean>
    <sigma>0.05</sigma>
  </Normal>

  <Normal name = 'gammay_dst'>
    <upperBound>20</upperBound>
    <lowerBound>-20</lowerBound>
    <mean>0</mean>
    <sigma>15</sigma>
  </Normal>
</Distributions>

<!-- SAMPLERS -->
<Samplers>
  <Grid name = 'grd_vl_ql_smp_dpt'>
    <variable name = 'x1' >
      <distribution>x1_dst</distribution>
      <grid type = 'value' construction = 'equal' steps = '10' upperBound = '10'>2</grid>
    </variable>
    <variable name='x2' >
      <distribution>x2_dst</distribution>
      <grid type = 'value' construction = 'equal' steps = '10' upperBound = '10'>2</grid>
    </variable>
    <variable name='gammay' >
      <distribution>gammay_dst</distribution>
      <grid type = 'value' construction = 'equal' steps = '10' lowerBound = '-20'>4</grid>
    </variable>
  </Grid>

  <Adaptive name = 'dpt_smp' verbosity='debug'>
    <ROM              class = 'Models'    type = 'ROM'           >accelerated_ROM</ROM>
    <Function         class = 'Functions' type = 'External'      >g_zr</Function>
    <TargetEvaluation class = 'DataObjects'     type = 'PointSet'  >nt_phy_dpt_dt</TargetEvaluation>
    <Convergence limit = '3000' forceIteration = 'False' weight = 'none' persistence = '5'>1e-2</Convergence>
      <variable name = 'x1'>
        <distribution>x1_dst</distribution>
      </variable>
      <variable name = 'x2'>
        <distribution>x2_dst</distribution>
      </variable>
      <variable name = 'gammay'>
        <distribution>gammay_dst</distribution>
      </variable>
  </Adaptive>
</Samplers>

<!-- MODELS -->
<Models>
  <ExternalModel name = 'xtr_mdl' subType = '' ModuleToLoad = 'SafestPointPP/safest_point_test_xtr_mdl'>
    <variables>x1,x2,gammay,g</variables>
  </ExternalModel>

  <ROM name = 'accelerated_ROM' subType = 'SciKitLearn'>
    <Features>x1,x2,gammay</Features>
    <Target>g_zr</Target>
    <SKLtype>svm|SVC</SKLtype>
    <kernel>rbf</kernel>
    <gamma>10</gamma>
    <tol>1e-5</tol>
    <C>50</C>
  </ROM>

  <PostProcessor name='SP' subType='SafestPoint'>
    <!-- List of Objects (external with respect to this PP) needed by this post-processor -->
    <Distribution     class = 'Distributions'  type = 'Normal'>x1_dst</Distribution>
    <Distribution     class = 'Distributions'  type = 'Normal'>x2_dst</Distribution>
    <Distribution     class = 'Distributions'  type = 'Normal'>gammay_dst</Distribution>
    <!- end of the list -->
    <controllable>
    	<variable name = 'x1'>
    		<distribution>x1_dst</distribution>
    		<grid type = 'value' steps = '20'>1</grid>
    	</variable>
    	<variable name = 'x2'>
    		<distribution>x2_dst</distribution>
    		<grid type = 'value' steps = '20'>1</grid>
    	</variable>
    </controllable>
    <non-controllable>
    	<variable name = 'gammay'>
    		<distribution>gammay_dst</distribution>
    		<grid type = 'value' steps = '20'>2</grid>
    	</variable>
    </non-controllable>
  </PostProcessor>
</Models>

<!-- FUNCTIONS -->
<Functions>
  <External name='g_zr' file='SafestPointPP/safest_point_test_g_zr.py'>
    <variable>g</variable>
  </External>
</Functions>

<!-- OUT-STREAMS -->
<OutStreams>
  <Print name = 'lmt_srf_dmp'>
  	<type>csv</type>
  	<source>lmt_srf_dt</source>
  </Print>

  <Print name = 'sfs_pnt_dmp'>
  	<type>csv</type>
  	<source>sfs_pnt_dt</source>
  </Print>
</OutStreams>

</Simulation>
\end{lstlisting}
The Python \textit{ExternalModel} is reported below:
\begin{lstlisting}[language=python]
def run(self,Input):
  self.g = self.x1+4*self.x2-self.gammay
\end{lstlisting}
The ``Goal Function'',the function that defines the transitions with respect the input space coordinates, is as follows:
\begin{lstlisting}[language=python]
def __residuumSign(self):
  if self.g<0 : return  1
  else        : return -1
\end{lstlisting}

%%%%% EXAMPLE 3
%\subsection{Example3}
%\label{subsec:ex1}
%example 3



    % ---------------------------------------------------------------------- %
    % References
    %
    \clearpage
    % If hyperref is included, then \phantomsection is already defined.
    % If not, we need to define it.
    \providecommand*{\phantomsection}{}
    \phantomsection
    \addcontentsline{toc}{section}{References}
    \bibliographystyle{ieeetr}
    \bibliography{raven_user_guide}


    % ---------------------------------------------------------------------- %
    %

    % \printindex

    %\include{distribution}

\end{document}
