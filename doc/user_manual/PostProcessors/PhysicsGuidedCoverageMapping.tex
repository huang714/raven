\label{PhysicsGuidedCoverageMapping}
to evaluate the uncertainty reduction fraction when using Feature to validate Target. As an algorithm of validation post-processor, PCM requirements to sub-nodes in \textit{Validation} PP include:

\begin{itemize}
  \item \xmlNode{Features}, \xmlDesc{comma separated string, required field}, specifies the names of the features.
  \item \xmlNode{Targets}, \xmlDesc{comma separated string, required field}, contains a comma separated list of
     targets. \nb Each target will be validated using all features listed in xml node \xmlNode{Features}. The
    number of targets is not necessarily equal to the number of features.
  \item \xmlNode{Measurements}, \xmlDesc{comma separated string, required field}, contains a comma separated list of
     measurements of the features. \nb Each measurement correspond to a feature listed in xml node \xmlNode{Features}. The
    number of measurements should be equal to the number of features and in the same order as the features listed in \xmlNode{Features}.
\end{itemize}

The output of PCM is comma separated list of strings in the format of "pri\_post\_stdReduct\_[targetName]", where [targetName] is the variable name target specified after "|" in \xmlNode{Targets}.


\textbf{Example:}
\begin{lstlisting}[style=XML,morekeywords={subType}]
<Simulation>
  ...
  <Models>
    ...
    <PostProcessor name="pcm" subType="PhysicsGuidedCoverageMapping">
      <Features>outputDataMC1|F1,outputDataMC1|F2</Features>
      <Targets>outputDataMC2|F2,outputDataMC2|F3,outputDataMC2|F4</Targets>
      <Measurements>msrData|F1,msrData|F2</Measurements>
    </PostProcessor>
    ...
  <Models>
  ...
<Simulation>
\end{lstlisting}
