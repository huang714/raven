
\section{Metrics}
\label{sec:Metrics}

\newcommand{\metrictypeI}[3]
{
  This metric interface directly with the metric available within \textit{#1}.
  The specifications of this metric must be defined within the XML block \xmlNode{#2}.
  This XML node needs to contain the following subnode:

  \begin{itemize}
    \item \xmlNode{metricType}\texttt{#3}\xmlNode{/metricType}, \xmlDesc{vertical bar (\texttt{|}) separated
      string, required field}.
  \end{itemize}

}
\newcommand{\metrictypeII}[3]
{
  \metrictypeI{#1}{#2}{#3}

  In addition to this XML subnode, the users can also specify the weights for given metric:
  \begin{itemize}
    \item \xmlNode{w}, \xmlDesc{comma separated floats, optional parameter}, the weights for each value in \textit{u}
      and \textit{v}. Default is None, which gives each value a weight of 1.0.
  \end{itemize}
}

\newcommand{\metrictypeIII}[3]
{
  \metrictypeI{#1}{#2}{#3}

  In addition to this XML subnode, the users can also specify the weights for given metric:
  \begin{itemize}
    \item \xmlNode{sample\_weight}, \xmlDesc{comma separated floats, optional parameter}, the weights for each value in \textit{u}
      and \textit{v}. Default is None, which gives each value a weight of 1.0.
  \end{itemize}
}

The Metrics block allows the user to specify the similarity/dissimilarity metrics to be used for other
RAVEN entities, such as \textbf{PostProcessors}, and \textbf{HybridModel}.

In the RAVEN input file these metrics are defined as follows:
\begin{lstlisting}[style=XML]
<Simulation>
  ...
  <Metrics>
    ...
    <Metric name='metricName' subType="MetricID">
      ...
     <param1>value</param1>
      ...
    </Metric>
    ...
  </Metrics>
  ...
</Simulation>
\end{lstlisting}

The metrics available in RAVEN can be categorized into several main classes:
\begin{itemize}
  \item \textbf{Paired Distance Metric}, distance metrics between two variables $u$ and $v$, such as \xmlString{euclidean},
    \xmlString{manhattan}, \xmlString{minkowski} and so on.
  \item \textbf{Regression Metric}, measure the regression performance, such as \xmlString{mean\_squared\_error},
    \xmlString{r2\_score}, \xmlString{explained\_variance\_score} and \xmlString{mean\_absolute\_error}.
  \item \textbf{Boolean Metric}, distance metrics between two boolean variables $u$ and $v$, such as
    \xmlString{dice}, \xmlString{hamming}, \xmlString{yule} and so on.
  \item \textbf{Pairwise Metric}, compute the distance or kernel between each pair of the two collections of input
    or observations in n-dimensional space.
    \nb These metrics can be only used in the clustering post-processor of data mining.
  \item \textbf{Other metric}, such as \xmlString{DTW}.
\end{itemize}

The valid \textbf{MetricID}s are: \xmlAttr{SKL}, \xmlAttr{ScipyMetric}, \xmlAttr{DTW}, \xmlAttr{CDFAreaDifference}, \xmlAttr{PDFCommonArea}, and \xmlAttr{DSS}. This XML node requires the following attributes:
\begin{itemize}
  \item \xmlAttr{name}, \xmlDesc{required string attribute}, user-defined name of this metric. \nb As with other
    objects, this name can be used to refer to this specific entity from other input blocks in the XML.
  \item \xmlAttr{subType}, \xmlDesc{required string attribute}, the desired type of Metric to use.
\end{itemize}

\nb If you are using \xmlAttr{ScipyMetric}, please pay more attention on the weight associated with the metric
calculations. Scipy does not normalize the weight during the calculation, and the results can be significant difference
from the normalized weight.

In RAVEN, lots of metrics are interfaces directly coupled with metrics available within \textbf{Scipy} and
\textbf{SciKit-Learn}. In this case, the algorithm for the metrics is choosen by the subnode \xmlNode{metricType}
under the parent node with \xmlAttr{subType} of either \xmlString{SKL} (metric from SciKit-Learn) or
\xmlString{ScipyMetric} (metric from Scipy). For
example, \xmlNode{metricType}\xmlString{paired\_distance|euclidean}\xmlNode{/metricType}.

In the following sub-sections, the input requirements for all of the metrics are presented in the following sections.

%%%%%%%%%%%%%%%%%%%%%%%%%%%%%%%%%%%%%%%%%%%%%%%%%%%%%%%%%%%%%%%%%%%%%%%%%%%%%%%%%%%%%%%%%%%%%%%%%%%
% paired_distance metric
%%%%%%%%%%%%%%%%%%%%%%%%%%%%%%%%%%%%%%%%%%%%%%%%%%%%%%%%%%%%%%%%%%%%%%%%%%%%%%%%%%%%%%%%%%%%%%%%%%%
\subsection{Paired Distance Metric}
\label{subsection:pairedDistance}

\subsubsection{Euclidean}
This metric compute the paired euclidean distances between $u$ and $v$, i.e.
\begin{equation}
  {||u-v||}_2
\end{equation}

\metrictypeI{SciKit-Learn}{SKL}{paired\_distance|euclidean}

\subsubsection{Cosine}
This metric computes the paired cosine distances between $u$ and $v$, i.e.
\begin{equation}
  1 - \frac{u \cdot v}{||u||_2 ||v||_2}
\end{equation}
where $u \cdot v$ is the dot product of $u$ and $v$

\metrictypeI{SciKit-Learn}{SKL}{paired\_distance|cosine}

\subsubsection{Manhattan}
This metric computes the L1 distances between $u$ and $v$, i.e.
\begin{equation}
  \sum_i {\left| u_i - v_i \right|}
\end{equation}

\metrictypeI{SciKit-Learn}{SKL}{paired\_distance|manhattan}

\subsubsection{Braycurtis}
This metric computes the Bray-Curtis distances between $u$ and $v$, i.e.
\begin{equation}
  \sum{|u_i-v_i|} / \sum{|u_i+v_i|}
\end{equation}
The Bray-Curtis distance is in the range $[0, 1]$.
\metrictypeII{Scipy}{ScipyMetric}{paired\_distance|braycurtis}

\subsubsection{Canberra}
This metric computes the Canberra distance between $u$ and $v$, i.e.
\begin{equation}
  d(u,v) = \sum_i \frac{|u_i-v_i|}{|u_i|+|v_i|}
\end{equation}

\metrictypeII{Scipy}{ScipyMetric}{paired\_distance|canberra}

\subsubsection{Correlation}
This metric computes the correlation distance between $u$ and $v$, i.e.
\begin{equation}
  1 - \frac{(u - \bar{u}) \cdot (v - \bar{v})}{{||(u - \bar{u})||}_2 {||(v - \bar{v})||}_2}
\end{equation}
where $\bar{u}$ is the mean of the elements of $u$

\metrictypeII{Scipy}{ScipyMetric}{paired\_distance|correlation}

\subsubsection{Minkowski}
This metric computes the Minkowski distance between $u$ and $v$, i.e.
\begin{equation}
  {||u-v||}_p = (\sum{|u_i - v_i|^p})^{1/p}
\end{equation}

\metrictypeII{Scipy}{ScipyMetric}{paired\_distance|minkowski}

\begin{itemize}
  \item \xmlNode{p}, \xmlDesc{float, required field}, value for the parameter $p$
\end{itemize}

In the RAVEN input file, these metrics are defined as follows:
\begin{lstlisting}[style=XML]
<Simulation>
  ...
  <Metrics>
    <Metric name="euclidean" subType="SKL">
        <metricType>paired_distance|euclidean</metricType>
    </Metric>
    <Metric name="cosine" subType="SKL">
        <metricType>paired_distance|cosine</metricType>
    </Metric>
    <Metric name="manhattan" subType="SKL">
        <metricType>paired_distance|manhattan</metricType>
    </Metric>
    <Metric name="braycurtis" subType="ScipyMetric">
      <metricType>paired_distance|braycurtis</metricType>
    </Metric>
    <Metric name="canberra" subType="ScipyMetric">
        <metricType>paired_distance|canberra</metricType>
    </Metric>
    <Metric name="correlation" subType="ScipyMetric">
        <metricType>paired_distance|correlation</metricType>
    </Metric>
    <Metric name="minkowski" subType="ScipyMetric">
        <metricType>paired_distance|minkowski</metricType>
        <p>5</p>
        <w>0.1, 0.1, 0.1, 0.1, 0.1</w>
    </Metric>
  </Metrics>
  ...
</Simulation>
\end{lstlisting}

%%%%%%%%%%%%%%%%%%%%%%%%%%%%%%%%%%%%%%%%%%%%%%%%%%%%%%%%%%%%%%%%%%%%%%%%%%%%%%%%%%%%%%%%%%%%%%%%%%%
% regression  metric
%%%%%%%%%%%%%%%%%%%%%%%%%%%%%%%%%%%%%%%%%%%%%%%%%%%%%%%%%%%%%%%%%%%%%%%%%%%%%%%%%%%%%%%%%%%%%%%%%%%
\subsection{Regression Metric}
\label{subsection:regression}

\subsubsection{Explained variance score}
This metric computes the explained variance regression score, i.e.
\begin{equation}
  1.0 - \frac{Var[u-v]}{Var[u]}
\end{equation}
The best possible score is 1.0, lower values are worse.

\metrictypeIII{Scikit-Learn}{SKL}{regression|explained\_variance\_score}

\subsubsection{Mean absolute error}
This metric computes mean absolute error, a risk metric corresponding to the expected value of the absolute
error loss or \textit{l1}-norm loss.
\begin{equation}
  \frac{1}{n_{samples}}\sum_{i=0}^{n_{samples}-1}|u_i-v_i|
\end{equation}

\metrictypeIII{Scikit-Learn}{SKL}{regression|mean\_absolute\_error}

\subsubsection{Mean squared error}
This metric computes mean square error, a risk metric corresponding to the expected value of the
squared error or loss.
\begin{equation}
  \frac{1}{n_{samples}}\sum_{i=0}^{n_{samples}-1}(u_i-v_i)^2
\end{equation}

\metrictypeIII{Scikit-Learn}{SKL}{regression|mean\_squared\_error}

\subsubsection{R2 score}
This metric computes the coefficient of determination, i.e.
\begin{equation}
  1.0-\frac{\sum_{i=0}^{n_{samples}-1}(u_i-v_i)^2}{\sum_{i=0}^{n_{samples}-1}(u_i-mean[u])^2}
\end{equation}
It provides a measure of how well future samples are likely to be predicted by the model.
Best possible score is 1.0 and it can be negative.

\metrictypeIII{Scikit-Learn}{SKL}{regression|r2\_score}

In the RAVEN input file, these metrics are defined as follows:
\begin{lstlisting}[style=XML]
<Simulation>
  ...
  <Metrics>
    <Metric name="explained_variance_score" subType="SKL">
        <metricType>regression|explained_variance_score</metricType>
        <sample_weight>0.1,0.1,0.1,0.05,0.05</sample_weight>
    </Metric>
    <Metric name="mean_absolute_error" subType="SKL">
        <metricType>regression|mean_absolute_error</metricType>
        <sample_weight>0.1,0.1,0.1,0.05,0.05</sample_weight>
    </Metric>
    <Metric name="r2_score" subType="SKL">
        <metricType>regression|r2_score</metricType>
        <sample_weight>0.1,0.1,0.1,0.05,0.05</sample_weight>
    </Metric>
    <Metric name="mean_squared_error" subType="SKL">
        <metricType>regression|mean_squared_error</metricType>
        <sample_weight>0.1,0.1,0.1,0.05,0.05</sample_weight>
    </Metric>
  </Metrics>
  ...
</Simulation>
\end{lstlisting}

%%%%%%%%%%%%%%%%%%%%%%%%%%%%%%%%%%%%%%%%%%%%%%%%%%%%%%%%%%%%%%%%%%%%%%%%%%%%%%%%%%%%%%%%%%%%%%%%%%%
% boolean  metric
%%%%%%%%%%%%%%%%%%%%%%%%%%%%%%%%%%%%%%%%%%%%%%%%%%%%%%%%%%%%%%%%%%%%%%%%%%%%%%%%%%%%%%%%%%%%%%%%%%%
\subsection{Boolean Metric}
\label{subsection:boolean}

\subsubsection{Dice}
This metric computes the Dice dissimilarity between two boolean variables $u$ and $v$
\begin{equation}
  \frac{c_{TF} + c_{FT}}{2c_{TT} + c_{FT} + c_{TF}}
\end{equation}
where $c_{ij}$ is the number of occurrences of $\mathtt{u[k]} = i$ and $\mathtt{v[k]} = j$ for $k < n$

\metrictypeII{Scipy}{ScipyMetric}{boolean|dice}

\subsubsection{Hamming}
This metric computes the Hamming distance between two boolean variables $u$ and $v$, i.e.
\begin{equation}
  \frac{c_{01} + c_{10}}{n}
\end{equation}
where $c_{ij}$ is the number of occurrences of $\mathtt{u[k]} = i$ and $\mathtt{v[k]} = j$ for $k < n$

\metrictypeII{Scipy}{ScipyMetric}{boolean|hamming}

\subsubsection{Jaccard}
This metric computes the Jaccard-Needham dissimilarity distance between two boolean variables $u$ and $v$, i.e.
\begin{equation}
  \frac{c_{TF} + c_{FT}}{c_{TT} + c_{FT} + c_{TF}}
\end{equation}
where $c_{ij}$ is the number of occurrences of $\mathtt{u[k]} = i$ and $\mathtt{v[k]} = j$ for $k < n$

\metrictypeII{Scipy}{ScipyMetric}{boolean|jaccard}

\subsubsection{Kulsinski}
This metric computes the Kulsinski dissimilarity distance between two boolean variables $u$ and $v$, i.e.
\begin{equation}
  \frac{c_{TF} + c_{FT} - c_{TT} + n}{c_{FT} + c_{TF} + n}
\end{equation}
where $c_{ij}$ is the number of occurrences of $\mathtt{u[k]} = i$ and $\mathtt{v[k]} = j$ for $k < n$

\metrictypeII{Scipy}{ScipyMetric}{boolean|kulsinski}

\subsubsection{Rogerstanimoto}
This metric computes the Rogers-Tanimoto dissimilarity distance between two boolean variables $u$ and $v$, i.e.
\begin{equation}
  \frac{R}{c_{TT} + c_{FF} + R}
\end{equation}
where $c_{ij}$ is the number of occurrences of $\mathtt{u[k]} = i$ and $\mathtt{v[k]} = j$
for $k < n$ and $R = 2(c_{TF} + c_{FT})$

\metrictypeII{Scipy}{ScipyMetric}{boolean|rogerstanimoto}

\subsubsection{Russellrao}
This metric computes the Russell-Rao dissimilarity distance between two boolean variables $u$ and $v$, i.e.
\begin{equation}
  \frac{n - c_{TT}}{n}
\end{equation}
where $c_{ij}$ is the number of occurrences of $\mathtt{u[k]} = i$ and $\mathtt{v[k]} = j$ for $k < n$

\metrictypeII{Scipy}{ScipyMetric}{boolean|russellrao}

\subsubsection{Sokalmichener}
This metric computes the Sokal-Michener dissimilarity distance between two boolean variables $u$ and $v$, i.e.
\begin{equation}
  \frac{R}{S + R}
\end{equation}
where $c_{ij}$ is the number of occurrences of $\mathtt{u[k]} = i$ and $\mathtt{v[k]} = j$ for
$k < n$, $R = 2 * (c_{TF} + c_{FT})$ and $S = c_{FF} + c_{TT}$

\metrictypeII{Scipy}{ScipyMetric}{boolean|sokalmichener}

\subsubsection{Sokalsneath}
This metric computes the Sokal-Sneath dissimilarity distance between two boolean variables $u$ and $v$, i.e.
\begin{equation}
  \frac{R}{c_{TT} + R}
\end{equation}
where $c_{ij}$ is the number of occurrences of $\mathtt{u[k]} = i$ and $\mathtt{v[k]} = j$
for $k < n$ and $R = 2(c_{TF} + c_{FT})$

\metrictypeII{Scipy}{ScipyMetric}{boolean|sokalsneath}

\subsubsection{Yule}
This metric computes the Yule dissimilarity distance between two boolean variables $u$ and $v$, i.e.
\begin{equation}
  \frac{R}{c_{TT} * c_{FF} + \frac{R}{2}}
\end{equation}
where $c_{ij}$ is the number of occurrences of $\mathtt{u[k]} = i$ and $\mathtt{v[k]} = j$ for
$k < n$ and $R = 2.0 * c_{TF} * c_{FT}$

\metrictypeII{Scipy}{ScipyMetric}{boolean|yule}

% TODO: the following metrics require different function interfaces
%  \item From scipy.spatial.distance
%       and $x \cdot y$ is the dot product of $x$ and $y$
%       \item mahalanobis: $\sqrt{ (u-v) V^{-1} (u-v)^T }$ where $V$ is the covariance matrix.  Note that the argument $VI$ is the inverse of $V$
%       \item seuclidean: $\sqrt{\sum {(u_i-v_i)^2 / V[x_i]}}$ where $V$ is the variance vector; $V[i]$ is the variance computed over all the i'th components of the points.
%        If not passed, it is automatically computed.
%       \item sqeuclidean: ${||u-v||}_2^2$

An example of Boolean metric defined in RAVEN is provided below:
\begin{lstlisting}[style=XML]
<Simulation>
  ...
  <Metrics>
    ...
    <Metric name="rogerstanimoto" subType="ScipyMetric">
        <metricType>boolean|rogerstanimoto</metricType>
    </Metric>
    <Metric name="dice" subType="ScipyMetric">
        <metricType>boolean|dice</metricType>
    </Metric>
    <Metric name="hamming" subType="ScipyMetric">
        <metricType>boolean|hamming</metricType>
    </Metric>
    <Metric name="jaccard" subType="ScipyMetric">
        <metricType>boolean|jaccard</metricType>
    </Metric>
    <Metric name="kulsinski" subType="ScipyMetric">
        <metricType>boolean|kulsinski</metricType>
    </Metric>
    <Metric name="russellrao" subType="ScipyMetric">
        <metricType>boolean|russellrao</metricType>
    </Metric>
    <Metric name="sokalmichener" subType="ScipyMetric">
        <metricType>boolean|sokalmichener</metricType>
    </Metric>
    <Metric name="sokalsneath" subType="ScipyMetric">
        <metricType>boolean|sokalsneath</metricType>
    </Metric>
    <Metric name="yule" subType="ScipyMetric">
        <metricType>boolean|yule</metricType>
    </Metric>
    ...
  </Metrics>
  ...
</Simulation>
\end{lstlisting}

%%%%%%%%%%%%%%%%%%%%%%%%%%%%%%%%%%%%%%%%%%%%%%%%%%%%%%%%%%%%%%%%%%%%%%%%%%%%%%%%%%%%%%%%%%%
\subsection{Dynamic Time Warping}
\label{subsection:DTW}
The Dynamic Time Warping (DTW) is a distance metrice that is used to measure similarity
between two sequences, i.e. temporal sequences.

The specifications of a DTW distance must be defined within the \xmlNode{Metric} XML block.

This XML node needs to contain the attributes:


\begin{itemize}
  \item \xmlNode{order},          \xmlDesc{int, required field},    order of the DTW calculation: $0$ specifies a classical DTW calculation and $1$ specifies
                                                                    a derivative DTW calculation
  \item \xmlNode{localDistance},  \xmlDesc{string, required field}, the ID of the distance function to be employed to determine the local distance
                                                                    evaluation of two time series. Available options are provided by the Scipy
                                                                    pairwise distances (cityblock, cosine, euclidean, $l1$, $l2$, manhattan,
                                                                    braycurtis, canberra, chebyshev, correlation, dice, hamming, jaccard,
                                                                    kulsinski, mahalanobis, matching, minkowski, rogerstanimoto, russellrao,
                                                                    seuclidean, sokalmichener, sokalsneath, sqeuclidean, yule)
\end{itemize}

An example of Minkowski distance defined in RAVEN is provided below:
\begin{lstlisting}[style=XML]
<Simulation>
  ...
  <Metrics>
    ...
    <Metric name="example" subType="DTW">
      <order>0</order>
      <localDistance>euclidean</localDistance>
    </Metric>
    ...
  </Metrics>
  ...
</Simulation>
\end{lstlisting}

%%%%%%%%%%%%%%%%%%%%%%%%%%%%%%%%%%%%%%%%%%%%%%%%%%%%%%%%%%%%%%%%%%%%%%%%%%%%%%%%%%%%%%%%
\subsection{CDFAreaDifference}
\label{subsubsec:metric_CDFAreaDifference}
This calculates the difference in area between the two CDFs.  This
metric supports using distributions as input.  Other inputs are
converted to a CDF.

\begin{equation}
  \text{CDF area difference} = \int_{-\infty}^{\infty}{\|CDF_a(x)-CDF_b(x)\|dx}
\end{equation}

This metric has the same units as $x$.  The closer the number is
to zero, the closer the match.  A perfect match would be 0.0.

An example is provided below:
\begin{lstlisting}[style=XML]
<Simulation>
  ...
  <Metrics>
    ...
    <Metric name="cdf_diff" subType="CDFAreaDifference"/>
    ...
  </Metrics>
  ...
</Simulation>
\end{lstlisting}

%%%%%%%%%%%%%%%%%%%%%%%%%%%%%%%%%%%%%%%%%%%%%%%%%%%%%%%%%%%%%%%%%%%%%%%%%%%%%%%%%%%%%%%%
\subsection{PDFCommonArea}
\label{subsubsec:metric_PDFCommonArea}
This calculates the common area between the two PDFs.  The higher the
value the closer the PDFs are.  This metric supports distributions as
inputs.  Other inputs are converted to a PDF.

\begin{equation}
  \text{PDF common area} = \int_{-\infty}^{\infty}{\min(PDF_a(x),PDF_b(x))}dx
\end{equation}

A perfect match would be 1.0.


An example is provided below:
\begin{lstlisting}[style=XML]
<Simulation>
  ...
  <Metrics>
    ...
    <Metric name="pdf_area" subType="PDFCommonArea"/>
    ...
  </Metrics>
  ...
</Simulation>
\end{lstlisting}

%%%%%%%%%%%%%%%%%%%%%%%%%%%%%%%%%%%%%%%%%%%%%%%%%%%%%%%%%%%%%%%%%%%%%%%%%%%%%%%%%%%%%%%%%%%%%%%%%%%
% Pairwise metric
%%%%%%%%%%%%%%%%%%%%%%%%%%%%%%%%%%%%%%%%%%%%%%%%%%%%%%%%%%%%%%%%%%%%%%%%%%%%%%%%%%%%%%%%%%%%%%%%%%%
\subsection{Pairwise Metric}
\label{subsection:pairwiseMetric}
This calculates the pairwise distance or kernel between each row of the two collections of inputs. This
metric can be only used in the \textbf{DataMining} post-processor.

\subsubsection{Polynomial}
Compute the polynomial kernel between $X$ and $Y$:
\begin{equation}
  K(X, Y) = (gamma <X, Y> + coef0)^{degree}
\end{equation}
\metrictypeI{Scikit-Learn}{PairwiseMetric}{kernel|Polynomial}
In addition to this XML subnode, the users can also specify the following subnodes:
\begin{itemize}
  \item \xmlNode{degree}, \xmlDesc{integer, optional parameter}, default `3'
  \item \xmlNode{gamma}, \xmlDesc{float, optional parameter}, default $1.0/numberColumnsInX$
  \item \xmlNode{coef0}, \xmlDesc{integer, optional parameter}, default `1'
\end{itemize}

\subsubsection{additive\_chi2}
Computes the additive chi-squared kernel between observations in $X$ and $Y$
The chi-squared kernel is computed between each pair of rows in $X$ and $Y$. $X$ and $Y$ have to be non-negative.
This kernel is most commonly applied to histograms.
The chi-squared kernel is given by:
\begin{equation}
  K(x, y) = - Sum [(x - y)^2 / (x + y)]
\end{equation}

\metrictypeI{Scikit-Learn}{PairwiseMetric}{kernel|additive\_chi2}

\subsubsection{chi2}
Computes the exponential chi-squared kernel between observations in $X$ and $Y$
The chi-squared kernel is computed between each pair of rows in $X$ and $Y$. $X$ and $Y$ have to be non-negative.
This kernel is most commonly applied to histograms.
The chi-squared kernel is given by:
\begin{equation}
  K(x, y) = exp(- gamma * Sum [(x - y)^2 / (x + y)])
\end{equation}

\metrictypeI{Scikit-Learn}{PairwiseMetric}{kernel|chi2}
In addition to this XML subnode, the users can also specify the following subnodes:
\begin{itemize}
  \item \xmlNode{gamma}, \xmlDesc{float, optional parameter}, default `1'
\end{itemize}

\subsubsection{cosine\_similarity}
Compute the cosine similarity between $X$ and $Y$:
\begin{equation}
  K(X, Y) = <X, Y> / (||X||*||Y||)
\end{equation}

\metrictypeI{Scikit-Learn}{PairwiseMetric}{kernel|cosine\_similarity}

\subsubsection{laplacian}
Computes the laplacian kernel between observations in $X$ and $Y$
The laplacian kernel is given by:
\begin{equation}
  K(x, y) = exp(- gamma * ||x - y||_1)
\end{equation}
for each pair of rows $x$ in $X$ and $y$ in $Y$.
\metrictypeI{Scikit-Learn}{PairwiseMetric}{kernel|laplacian}
In addition to this XML subnode, the users can also specify the following subnodes:
\begin{itemize}
  \item \xmlNode{gamma}, \xmlDesc{float, optional parameter}, default $1.0/numberColumnsInX$
\end{itemize}

\subsubsection{linear}
computes the linear kernel between $X$ and $Y$
\begin{equation}
  K(X, Y) = X^T * Y
\end{equation}
\metrictypeI{Scikit-Learn}{PairwiseMetric}{kernel|linear}

\subsubsection{rbf}
Computes the laplacian kernel between observations in $X$ and $Y$
The laplacian kernel is given by:
\begin{equation}
  K(x, y) = exp(- gamma * ||x - y||^2)
\end{equation}
for each pair of rows $x$ in $X$ and $y$ in $Y$.
\metrictypeI{Scikit-Learn}{PairwiseMetric}{kernel|rbf}
\begin{itemize}
  \item \xmlNode{gamma}, \xmlDesc{float, optional parameter}, default $1.0/numberColumnsInX$
\end{itemize}

\subsubsection{sigmoid}
Compute the sigmoid kernel between $X$ and $Y$:
\begin{equation}
  K(X, Y) = tanh(gamma <X, Y> + coef0)
\end{equation}
\metrictypeI{Scikit-Learn}{PairwiseMetric}{kernel|sigmoid}
\begin{itemize}
  \item \xmlNode{gamma}, \xmlDesc{float, optional parameter}, default $1.0/numberColumnsInX$
  \item \xmlNode{coef0}, \xmlDesc{integer, optional parameter}, default `1'
\end{itemize}

\subsubsection{Distance Based Metric}

\metrictypeI{Scipy or Scikit-Learn}{PairwiseMetric}{pairwise|`metric'}

\nb \textbf{`metric'}, the distance metric to use, this can be `braycurtis', `canberra', `chebyshev', `correlation',
`cosine', `dice', `euclidean', `hamming', `jaccard', `kulsinski', `matching', `minkowski', `rogerstanimoto',
`russellrao', `sokalmichener', `sokalsneath', `yule', `manhatten'. The definition for each metric can be found
in previous sections.

In addition to this XML subnode, the users can also specify the corresponding parameters for each `metric' according to
previous sections.

%%%%%%%%%%%%%%%%%%%%%%%%%%%%%%%%%%%%%%%%%%%%%%%%%%%%%%%%%%%%%%%%%%%%%%%%%%%%%%%%%%%%%%%%%%%
\subsection{Dynamical System Scaling}
\label{subsection:DSS}
The Dynamical System Scaling (DSS) is a distance metrics that is used to measure the separation
between two time-dependent data sets.

The specifications of a DSS metric is defined within the \xmlNode{Metric} XML block. The XML node \xmlAttr{subType} must be \textbf{PPDSS} (see \ref{subsubsec:Validation}) in the \xmlNode{PostProcessor} for the outputs of the post-processor to be in the right format for DSS metric inputs.

An example of DSS  defined in RAVEN is provided below:
\begin{lstlisting}[style=XML]
<Simulation>
  ...
  <Metrics>
    ...
    <Metric name="example" subType="DSS">
    </Metric>
    ...
  </Metrics>
  ...
</Simulation>
\end{lstlisting}
