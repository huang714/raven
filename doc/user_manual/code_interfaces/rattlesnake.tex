%%%%%%%%%%%%%%%%%%%%%%%%%%%%%%%%%%%%%%%%%%%%%%%%%
%%%%%%%%%%%%% RATTLESNAKE INTERFACE %%%%%%%%%%%%%
%%%%%%%%%%%%%%%%%%%%%%%%%%%%%%%%%%%%%%%%%%%%%%%%%
\subsection{Rattlesnake Interfaces} \label{RattlesnakeInterfaces}
%
This section covers the input specification for running Rattlesnake through RAVEN. It is important
to notice that this short explanation assumes that the reader already knows how to use Rattlesnake.
The interface can be used to perturb the Rattlesnake MOOSE-based input file as well as the Yak
cross section libraries XML input files (e.g., multigroup cross section libraries) and Instant format
cross section libraries.
%
%%%%%%%%%%%%%%%%%%%%%%%%%%%%%%%%%%%%%%%%%%%%%%%%%%
\subsubsection{Files}
\xmlNode{Files} works the same as in other interfaces with name and type
attributes for each node entry.  The \xmlAttr{name} attribute is a user-chosen internal
name for the file contained in the node, and \xmlAttr{type} identifies which base-level
interface the file is used within.  \xmlAttr{type} should only be specified for inputs
that RAVEN will perturb. Take Rattlesnake input files for example, \xmlAttr{type} should
be \xmlString{RattlesnakeInput}.

\paragraph{Perturb Yak Multigroup Cross Section Libraries}
If the user would like to perturb the Yak multigroup cross section libraries, the user need to use the
\xmlString{YakXSInput} for the \xmlAttr{type} of the libraries. In addition, the \xmlAttr{type} of the
alias files that are used to perturb the Yak multigroup cross section libraries should be
\xmlString{YakXSAliasInput}. The following is an example of a typical \xmlNode{Files} block.
%
\begin{lstlisting}[style=XML]
<Files>
  <Input name='rattlesnakeInput' type='RattlesnakeInput'>simple_diffusion.i</Input>
  <Input name='crossSection'    type='YakXSInput'>xs.xml</Input>
  <Input name='alias' type='YakXSAliasInput'>alias.xml</Input>
</Files>
\end{lstlisting}
%
The alias files are employed to define the variables that will be used to perturb Yak multigroup cross section
libraries. The following is an example of a typical alias file:
%
\begin{lstlisting}[style=XML]
<Multigroup_Cross_Section_Libraries Name="twigl" NGroup="2" Type="rel">
    <Multigroup_Cross_Section_Library ID="1">
        <Fission gridIndex="1" mat="pseudo-seed1" gIndex="1">f11</Fission>
        <Capture gridIndex="1" mat="pseudo-seed1" gIndex="1">c11</Capture>
        <TotalScattering gridIndex="1" mat="pseudo-seed1" gIndex="1">t11</TotalScattering>
        <Nu gridIndex="1" mat="pseudo-seed1" gIndex="1">n11</Nu>
        <Fission gridIndex="1" mat="pseudo-seed2" gIndex="2">f22</Fission>
        <Capture gridIndex="1" mat="pseudo-seed2" gIndex="2">c22</Capture>
        <TotalScattering gridIndex="1" mat="pseudo-seed2" gIndex="2">t22</TotalScattering>
        <Nu gridIndex="1" mat="pseudo-seed2" gIndex="2">n22</Nu>
        <Fission gridIndex="1" mat="pseudo-seed1-dup" gIndex="1">f11</Fission>
        <Capture gridIndex="1" mat="pseudo-seed1-dup" gIndex="1">c11</Capture>
        <TotalScattering gridIndex="1" mat="pseudo-seed1-dup" gIndex="1">t11</TotalScattering>
        <Nu gridIndex="1" mat="pseudo-seed1-dup" gIndex="1">n11</Nu>
        <Transport gridIndex="1" mat="pseudo-seed1-dup" gIndex="1">d11</Transport>
    </Multigroup_Cross_Section_Library>
</Multigroup_Cross_Section_Libraries>
\end{lstlisting}
%
In the above alias file, the \xmlAttr{Name} of \xmlNode{Multigroup\_Cross\_Section\_Libraries} are used to indicate
which Yak multigroup cross section library input file will be perturbed.
The \xmlAttr{NGroup}, \xmlAttr{ID}, and \xmlNode{Multigroup\_Cross\_Section\_Library}
should be consistent with the Yak multigroup cross section library input files.
The \xmlNode{Fission}, \xmlNode{Capture}, \xmlNode{TotalScattering}, \xmlNode{Nu}, \xmlAttr{gridIndex},
\xmlAttr{mat}, and \xmlAttr{gIndex} are used to find the corresponding cross sections in the Yak multigroup cross
section library input files. For example:
%
\begin{lstlisting}[style=XML]
<Fission gridIndex="1" mat="pseudo-seed1" gIndex="1">f11</Fission>
\end{lstlisting}
%
This node defines an alias with name \xmlString{f11} used to represent the fission cross section at energy group \xmlString{1}
for material with name 'pseudo-seed1' at grid index \xmlString{1} in the Yak multigroup cross section library input files.

\nb The attribute \xmlAttr{Type="rel"} indicates that the cross sections will be perturbed relatively (i.e. perturbed by
percents). In this case, the user also needs to specify a relative covariance matrix for \xmlNode{covaraince \xmlAttr{type="rel"}} in
\xmlNode{MultivariateNormal} distribution, and the values for \xmlNode{mu} should be `ones'. In the other case, if
the user choose \xmlAttr{Type="abs"}, the cross sections will be perturbed absolutely (i.e. perturbed by values), and
the user needs to provide an absolute covariance matrix and specify `zeros' for \xmlNode{mu} in \xmlNode{MultivariateNormal}
distribution.

\nb Currently, only the following cross sections can be perturbed by the user: Fission, Capture, Nu, TotalScattering,
and Transport.

\paragraph{Perturb Instant format Cross Section Libraries}
If the user would like to perturb the Instant cross section libraries, the user need to use the
\xmlString{InstantXSInput} for the \xmlAttr{type} of the libraries. In addition, the \xmlAttr{type} of the
alias files that are used to perturb the Instant format cross section libraries should be
\xmlString{InstantXSAliasInput}. The following is an example of a typical \xmlNode{Files} block.
%
\begin{lstlisting}[style=XML]
<Files>
  <Input name='rattlesnakeInput' type='RattlesnakeInput'>iaea2d_ls_sn.i</Input>
  <Input name='crossSection'    type='InstantXSInput'>iaea2d_materials.xml</Input>
  <Input name='alias' type='InstantXSAliasInput'>alias.xml</Input>
</Files>
\end{lstlisting}
%
The alias files are employed to define the variables that will be used to perturb Instant format cross section
libraries. The following is an example of a typical alias file:
%
\begin{lstlisting}[style=XML]
<Materials>
  <Macros NG="2" Type="rel">
    <material ID="1">
      <FissionXS gIndex="1">f11</FissionXS>
      <CaptureXS gIndex="1">c11</CaptureXS>
      <TotalScatteringXS gIndex="1">t11<TotalScatteringXS>
      <Nu gIndex="1">n11</Nu>
      <DiffusionCoefficient gIndex="1">d11</DiffusionCoefficient>
    </material>
  </Macros>
</Materials>
\end{lstlisting}

%
In the above alias file, the \xmlAttr{NG} and \xmlAttr{ID} should be consistent with the Instant format cross
section library input files. The \xmlNode{FissionXS}, \xmlNode{CaptureXS}, \xmlNode{TotalScatteringXS}, \xmlNode{Nu}, \xmlAttr{gIndex},
are used to find the corresponding cross sections in the Instant format cross
section library input files. For example, the variable \xmlString{f11} used to represent the fission cross section at energy group \xmlString{1}
for material with \xmlString{ID} equal \xmlString{1} in the given cross section library.

\nb The attribute \xmlAttr{Type="rel"} indicates that the cross sections will be perturbed relatively (i.e. perturbed by
percents). In this case, the user also needs to specify a relative covariance matrix for \xmlNode{covariance \xmlAttr{type="rel"}} in
\xmlNode{MultivariateNormal} distribution, and the values for \xmlNode{mu} should be `ones'. In the other case, if
the user choose \xmlAttr{Type="abs"}, the cross sections will be perturbed absolutely (i.e. perturbed by values), and
the user needs to provide an absolute covariance matrix and specify `zeros' for \xmlNode{mu} in \xmlNode{MultivariateNormal}
distribution.

\nb Currently, only the following cross sections can be perturbed by the user: FissionXS, CaptureXS, Nu, TotalScatteringXS,
and DiffusionCoefficient.

%%%%%%%%%%%%%%%%%%%%%%%%%%%%%%%%%%%%%%%%%%%%%%%%%%
\subsubsection{Models}
A user provides paths to executables and aliases for sampled variables within the
\xmlNode{Models} block.  The \xmlNode{Code} block will contain attributes \xmlNode{name} and
\xmlNode{subType}. The \xmlNode{name} identifies that particular \xmlNode{Code} model within RAVEN, and
\xmlNode{subType} specifies which code interface the model will use. The \xmlNode{executable}
block should contain the absolute or relative (with respect to the current working
directory) path to Rattlesnake that RAVEN will use to run generated input
files.

An example \xmlNode{Models} block follows.

\begin{lstlisting}[style=XML]
<Models>
  <Code name="Rattlesnake" subType="Rattlesnake">
    <executable>%FRAMEWORK_DIR%/../../rattlesnake/
     rattlesnake-%METHOD%</executable>
  </Code>
</Models>
\end{lstlisting}

%%%%%%%%%%%%%%%%%%%%%%%%%%%%%%%%%%%%%%%%%%%%%%%%%%
\subsubsection{Distributions}
The \xmlNode{Distributions} block defines all distributions used to
sample variables in the current RAVEN run.

For all the possible distributions and their possible inputs please
refer to the Distributions chapter (see~\ref{sec:distributions}).
%
It is good practice to name the distribution something similar to what kind of
variable is going to be sampled, since there might be many variables with the
same kind of distributions but different input parameters.

%%%%%%%%%%%%%%%%%%%%%%%%%%%%%%%%%%%%%%%%%%%%%%%%%%
\paragraph{Samplers}
The \xmlNode{Samplers} block defines the variables to be sampled.
After defining a sampling scheme, the variables to be sampled and
their distributions are identified in the \xmlNode{variable} blocks.
The name attribute in the \xmlNode{variable} block must either be the
full MooseBasedApp (Rattlesnake) model variable name, the alias name specifed in
\xmlNode{Models}, or the variable name specified in the provided alias files.

For listings of available samplers, please refer to the Samplers chapter (see~\ref{sec:Samplers}).
See the following for an example of a grid based sampler for
the first energy group fission and capture cross sections  (both of which have
defined in alias files provided in \xmlNode{Files}).

\begin{lstlisting}[style=XML]
<Samplers>
  <Grid name="Grid_sampling">
    <variable name="fission_group_1" >
      <distribution>fission_dist</distribution>
      <grid type="value" construction="custom">1.0 2.0</grid>
    </variable>
    <variable name="capture_group_1">
      <distribution>capture_dist</distribution>
      <grid type="value" construction="custom">3.0 6.0</grid>
    </variable>
  </Grid>
</Samplers>
\end{lstlisting}

%%%%%%%%%%%%%%%%%%%%%%%%%%%%%%%%%%%%%%%%%%%%%%%%%%
\subsubsection{Steps}
For a Rattlesnake interface, the \xmlNode{MultiRun} step type will most likely be used. First, the step needs
to be named: this name will be one of the names used in the \xmlNode{Sequence} block. In our example, \xmlString{Grid\_Rattlesnake}.
%
\begin{lstlisting}[style=XML]
<MultiRun name='Grid_Rattlesnake' verbosity='debug'>
    <Input   class='Files' type=''>RattlesnakeInput.i</Input>
    <Input   class='Files' type=''>xs.xml</Input>
    <Input   class='Files' type=''>alias.xml</Input>
    <Model   class='Models' type='Code'>Rattlesnake</Model>
    <Sampler class='Samplers' type='Grid'>Grid_Samplering</Sampler>
    <Output  class='DataObjects' type='PointSet'>solns</Ouput>
\end{lstlisting}
%
With this step, we need to import all the files needed for the simulation:
%
\begin{itemize}
  \item Rattlesnake MOOSE-based input file;
  \item Yak multigroup cross section libraries input files (XML);
  \item Yak alias files used to define the perturbed variables (XML).
\end{itemize}
We then need to define \xmlNode{Model}, \xmlNode{Sampler} and \xmlNode{Output}. The \xmlNode{Output} can be
\xmlNode{DataObjects} or \xmlNode{OutStreams}.

